% Section
\part{Docker install}
いつもここからhttps://www.docker.com/
%chapter
\section{Docker コンテナの作成}
以下でDockerコンテナを作成\\
\begin{verbatim}
  docker run -it centos:6 /bin/bash
\end{verbatim}


\part{LDAP Server}
\section{LDAP Serverのセットアップ}
\subsection{Dockerコンテナの起動}
docker プロセスにアタッチ
\begin{verbatim}
  docker start <CONTAINER ID>
  docker attach <CONTAINER ID>
\end{verbatim}
\subsection{LDAPのインストールと設定}
ldapユーザを作成
\begin{verbatim}
  useradd ldap
\end{verbatim}
OpenLDAPをインストール
\begin{verbatim}
  yum -y install openldap-servers openldap-clients
\end{verbatim}
OpenLDAPサーバーにデータを登録する際の「ldapadd」コマンドや検索用の「ldapsearch」コマンドは「openldap-clients」パッケージに入っている\\
そのため、OpenLDAPサーバーのみであっても、「openldap-clients」パッケージもインストールしておくこと\\
\newline
初期設定削除
\begin{verbatim}
  rm -rf /etc/openldap/slapd.d/*
  rm -rf /var/lib/ldap/*
  rm -rf /etc/openldap/slap.d
\end{verbatim}
設定ファイルのコピー
\begin{verbatim}
  cp -a /usr/share/openldap-servers/DB_CONFIG.example /var/lib/ldap/DB_CONFIG
  chown ldap /var/lib/ldap/DB_CONFIG
  cp -a /usr/share/openldap-servers/slapd.conf.obsolete /etc/openldap/slapd.conf
\end{verbatim}
マスターパスワード作成
\begin{verbatim}
  slappasswd
\end{verbatim}
ここで出力されたパスワードのHash値は控えておく\\
設定ファイルの編集
\begin{verbatim}
  vi /etc/openldap/slapd.conf 
\end{verbatim}
\begin{verbatim}
----------------------------------------------------------------------------
# スキーマファイル設定
include /etc/openldap/schema/corba.schema
include /etc/openldap/schema/core.schema
include /etc/openldap/schema/cosine.schema
include /etc/openldap/schema/duaconf.schema
include /etc/openldap/schema/dyngroup.schema
include /etc/openldap/schema/inetorgperson.schema
include /etc/openldap/schema/java.schema
include /etc/openldap/schema/misc.schema
include /etc/openldap/schema/nis.schema
include /etc/openldap/schema/openldap.schema
include /etc/openldap/schema/ppolicy.schema
include /etc/openldap/schema/collective.schema

# 接続プロトコル
allow bind_v2

# 管理ファイル
pidfile     /var/run/openldap/slapd.pid
argsfile    /var/run/openldap/slapd.args

# TLS設定
#TLSCACertificatePath  /etc/openldap/ssl/cacert.pem
#TLSCertificateFile    /etc/openldap/ssl/server.crt
#TLSCertificateKeyFile /etc/openldap/ssl/server.key

# userPasswordに関するアクセス権
access to attrs=userPassword
    by self write
    by dn="cn=Manager,dc=example,dc=com" write
    by anonymous auth
    by * none

# その他の属性に対するアクセス権
access to *
    by self write
    by dn="cn=Manager,dc=example,dc=com" write
    by * read

# monitorデータベースに対するアクセス権
database monitor
access to *
    by dn.exact="cn=Manager,dc=example,dc=com" read
    by * none

# データベース設定
database    bdb
suffix      "dc=example,dc=com"
checkpoint  1024 15
rootdn      "cn=Manager,dc=example,dc=com"
rootpw      {SSHA}your.pass.hash
directory   /var/lib/ldap

# indexの設定
index objectClass                       eq,pres
index ou,cn,mail,surname,givenname      eq,pres,sub
index uidNumber,gidNumber,loginShell    eq,pres
index uid,memberUid                     eq,pres,sub
index nisMapName,nisMapEntry            eq,pres,sub
----------------------------------------------------------------------------
\end{verbatim}
設定のテスト
\begin{verbatim}
  slaptest -u -v -f /etc/openldap/slapd.conf
\end{verbatim}
config file testing succeeded\\
/etc/openldap/slapd.d/ の更新
\begin{verbatim}
  slaptest -f /etc/openldap/slapd.conf -F /etc/openldap/slapd.d
\end{verbatim}
エラーが出るが無視する
\begin{verbatim}
bdb_db_open: database "dc=example,dc=com": db_open(/var/lib/ldap/id2entry.bdb) failed:
 No such file or directory (2).
backend_startup_one (type=bdb, suffix="dc=example,dc=com"): bi_db_open failed! (2)
slap_startup failed (test would succeed using the -u switch)
\end{verbatim}
LDAP起動
\begin{verbatim}
  service slapd start
  chkconfig slapd on
\end{verbatim}

\subsection{データの登録}
基本データの作成
\begin{verbatim}
  mkdir -p /etc/openldap/ldif
  vi /etc/openldap/ldif/base.ldif
\end{verbatim}
\begin{verbatim}
----------------------------------------------------------------------------
# ドメイン
dn: dc=example,dc=com
objectClass: dcObject
objectClass: organization
dc: example
o: Example co.,Ltd

# 管理者
dn: cn=Manager,dc=example,dc=com
objectClass: organizationalRole
cn: Manager

# システムユーザー
dn: ou=People,dc=example,dc=com
objectClass: organizationalUnit
ou: People

# システムグループ
dn: ou=Group,dc=example,dc=com
objectClass: organizationalUnit
ou: Group

# アドレス帳
dn: ou=Address,dc=example,dc=com
objectClass: organizationalUnit
ou: Address
----------------------------------------------------------------------------
\end{verbatim}
LDAPサーバーに基本データ登録
\begin{verbatim}
  ldapadd -x -D "cn=Manager,dc=example,dc=com" -W -f /etc/openldap/ldif/base.ldif
\end{verbatim}
パスワード入力を求められるため、「slappasswd」コマンドで入力したパスワードを入力すること。\\
情報システム部のグループデータ作成
\begin{verbatim}
  vi /etc/openldap/ldif/group.ldif
\end{verbatim}
\begin{verbatim}
----------------------------------------------------------------------------
# 情報システム部
dn: cn=system,ou=Group,dc=example,dc=com
objectClass: posixGroup
objectClass: top
cn: system
gidNumber: 1000
----------------------------------------------------------------------------
\end{verbatim}
情報システム部のグループデータ登録
\begin{verbatim}
  ldapadd -x -D "cn=Manager,dc=example,dc=com" -W -f /etc/openldap/ldif/group.ldif
\end{verbatim}
情報システム部のユーザーデータ作成
\begin{verbatim}
  vi /etc/openldap/ldif/user.ldif
\end{verbatim}
\begin{verbatim}
----------------------------------------------------------------------------
# 部長:武田 貴彦
dn: uid=takahiko.takeda,ou=People,dc=example,dc=com
objectClass: shadowAccount
objectClass: posixAccount
objectClass: account
objectClass: top
cn: Takahiko Takeda
uid: takahiko.takeda
uidNumber: 1001
gidNumber: 1000
homeDirectory: /home/takahiko.takeda
loginShell: /bin/bash
shadowMin: 0
shadowMax: 99999
shadowWarning: 7
shadowLastChange: 16175
userPassword: {SSHA}0IBfkSHm6eDVouPuwRlvEBBBofDUNul6

# 開発課 課長:横山 真也
dn: uid=shinya.yokoyama,ou=People,dc=example,dc=com
objectClass: shadowAccount
objectClass: posixAccount
objectClass: account
objectClass: top
cn: Shinya Yokoyama
uid: shinya.yokoyama
uidNumber: 1002
gidNumber: 1000
homeDirectory: /home/shinya.yokoyama
loginShell: /bin/bash
shadowMin: 0
shadowMax: 99999
shadowWarning: 7
shadowLastChange: 16175
userPassword: {SSHA}WjskiArncjkXcoLNsX49TE5+L6qBfeZL

# 運用課 課長:井上 修
dn: uid=osamu.inoue,ou=People,dc=example,dc=com
objectClass: shadowAccount
objectClass: posixAccount
objectClass: account
objectClass: top
cn: Osamu Inoue
uid: osamu.inoue
uidNumber: 1003
gidNumber: 1000
homeDirectory: /home/osamu.inoue
loginShell: /bin/bash
shadowMin: 0
shadowMax: 99999
shadowWarning: 7
shadowLastChange: 16175
userPassword: {SSHA}IMy6+0Zvb5mgsrByu0zrD68K9yHtFVR6

# 開発課 社員:石川 直樹
dn: uid=naoki.ishikawa,ou=People,dc=example,dc=com
objectClass: shadowAccount
objectClass: posixAccount
objectClass: account
objectClass: top
cn: Naoki Ishikawa
uid: naoki.ishikawa
uidNumber: 1004
gidNumber: 1000
homeDirectory: /home/naoki.ishikawa
loginShell: /bin/bash
shadowMin: 0
shadowMax: 99999
shadowWarning: 7
shadowLastChange: 16175
userPassword: {SSHA}1ixM0yQBZZlFaReiN778DKGSoo5peemc

# 運用課 社員:田村 和夫
dn: uid=kazuo.tamura,ou=People,dc=example,dc=com
objectClass: shadowAccount
objectClass: posixAccount
objectClass: account
objectClass: top
cn: Kazuo Tamura
uid: kazuo.tamura
uidNumber: 1005
gidNumber: 1000
homeDirectory: /home/kazuo.tamura
loginShell: /bin/bash
shadowMin: 0
shadowMax: 99999
shadowWarning: 7
shadowLastChange: 16175
userPassword: {SSHA}vfmOEDQbUqm43yaqFXIOwxMS9y15mfxt
----------------------------------------------------------------------------
\end{verbatim}
情報システム部のユーザーデータ登録
\begin{verbatim}
  ldapadd -x -D "cn=Manager,dc=example,dc=com" -W -f /etc/openldap/ldif/user.ldif
\end{verbatim}


%chapter
\section{節}
section
\subsection{小節}
subsection
\subsubsection{少々節}
subsubsection